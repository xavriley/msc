\documentclass[11pt]{article} % do not change this line
\input{BigDataStyle.txt}      % do not change this line
\usepackage{amsmath,amsfonts,amssymb,amsthm,latexsym,graphicx}

\emergencystretch=5mm
\tolerance=400
\allowdisplaybreaks[4]

\theoremstyle{plain}
\newtheorem{theorem}{Theorem}[section]
\newtheorem{proposition}[theorem]{Proposition}
\newtheorem{corollary}[theorem]{Corollary}
\newtheorem{lemma}[theorem]{Lemma}
\newtheorem{problem}[theorem]{Problem}

\theoremstyle{definition}
\newtheorem*{remark}{Remark}

% Can I make a more precise title than this?
\title{Distributed time synchronization in musical applications}
\author{Xavier Riley}

\newcommand{\Programme}{Distributed and Networked Systems}

\begin{document}
\maketitle

\declaration

% Motivations
% Artistic output but also, music industry is large and $$$
% requirements of a musical performance
% - ensemble
% - timekeeping (layers - a year is one orbit of the sun, a day one rotation on the earths' axis)
% - colocated as opposed to geo-distributed
% clock synchronization literature
% - internal, external, pulse based - how have these been applied to music so far?
% Ableton Live
% - aims, tradeoffs, assumptions (reliable network)
% - works via multicast, discovery, exchanging timelines
% - fault tolerance concepts (sending on all changes)
% Can it be improved upon without violating simplicity? To provide:
% - resilience to lost messages (reliable broadcast) - does it already do this?
% - resistance to crash faults (simple consensus) - protocol assumes a shared network - not sure this makes sense
% - improvements in synchronization bounds - what are the current limits? - are they perceptible? (within 10ms)
% - other things I haven't considered yet

% how to test it? One option - get several devices and record into same soundcard
% has the benefit of being a realistic test, easier to get setup
% other option - use soundflower and write to a multichannel wav - maybe more reproducible

% further ideas - even better internal sync (at the cost of bandwidth) - Google paper
% resettable oscillators - more naturally maps onto how humans synchronize perhaps
% options for geo-distributed time synchronization
% extensions beyond music to other time sensitive domains

\begin{abstract}
  Your abstract goes here.
\end{abstract}

\section{Section one}

This is a sample section.

\section{Section two}

An example of a reference:
\cite{hastie/etal:2009}.

\section{Section three}
\section{Section four}
\section{Section five}
\section{Section six}
\section{Section seven}

\bibliographystyle{plain}
\bibliography{bibliography}
\end{document}
