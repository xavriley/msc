\documentclass[11pt]{article} % do not change this line
\input{BigDataStyle.txt}      % do not change this line
\usepackage{amsmath,amsfonts,amssymb,amsthm,latexsym,graphicx}

\emergencystretch=5mm
\tolerance=400
\allowdisplaybreaks[4]

\theoremstyle{plain}
\newtheorem{theorem}{Theorem}[section]
\newtheorem{proposition}[theorem]{Proposition}
\newtheorem{corollary}[theorem]{Corollary}
\newtheorem{lemma}[theorem]{Lemma}
\newtheorem{problem}[theorem]{Problem}

\theoremstyle{definition}
\newtheorem*{remark}{Remark}

\title{Breaking Links: Evaluating distributed time synchronization for musical applications using Ableton Link}
\author{Xavier Riley}

\newcommand{\Programme}{Distributed and Networked Systems}

\begin{document}
\maketitle

\declaration

% assessment criteria
% - bounds on synchronization
% - message complexity /bandwidth
% - difficulty of setup/integration
% - failure modes
% - resilience to drift
% - topology / spof?

% Ableton Live
% - aims, tradeoffs, assumptions (reliable network)
% - works via multicast, discovery, exchanging timelines
% - fault tolerance concepts (sending on all changes)
% Can it be improved upon without violating simplicity? To provide:
% - resilience to lost messages (reliable broadcast) - does it already do this?
% - resistance to crash faults (simple consensus) - protocol assumes a shared network - not sure this makes sense
% - improvements in synchronization bounds - what are the current limits? - are they perceptible? (within 10ms)
% - other things I haven't considered yet

% Possible experiment criteria
% We can't measure things like dropped writes or stale reads so what are the
% criteria for a music system? Draw comparisons with gossip analysis -
% convergence time, stability, resilience to pertubations. Also check how clock
% sync literature handles this.
% The main thing is that the tempo stays in sync for as much time as possible

% further ideas - even better internal sync (at the cost of bandwidth) - Google paper
% resettable oscillators - more naturally maps onto how humans synchronize perhaps
% options for geo-distributed time synchronization
% extensions beyond music to other time sensitive domains

% What your project *must* contain
% motivations and original aims including how this work may help in future career
% assessment inc. self evaluation. How did it go? What did you do right or
%      wrong? What have you learnt about planning and execution? Where next?

% should contain
% - abstract
% - introduction
% - background research (literature survey/review)
% - either

% sofware product - software engineering method, requirements analysis, design,
% implementation, testing - also user or installation manual

% theoretical - development of theory, inc small programs, explanation of
% algorithms, descriptions of hard theory, results, analysis

% experimental - experimental results, analysis, conclusions

% - professional issues (as appendix - c. 1000 words)
% - self assessment
% - bibliography
% - any code, including instructions on how to run
% - other - layout diagrams, sample output, program listing

\begin{abstract}
  With the recent explosion of connected musical devices, the challenge of
  making these play "in time" with each other mounts against application
  developers and device manufacturers. Ableton Link\cite{goltz2018ableton} aims
  to provide robust, resilient musical synchronization using principles from
  distributed systems programming, in contrast to previous master/slave
  approaches. However the evaluation criteria for such a system are not well
  represented in the existing literature, with particular reference to a
  musical context.  The following presents a system for empirical testing of
  the Ableton Link library using the Jepsen\cite{jepsen} testing framework,
  along with a set of criteria for evaluting similar libraries that may be
  developed in future.
\end{abstract}

% Motivations
%
% Artistic output but also, music industry is large and $$$
% requirements of a musical performance
% - ensemble
% - timekeeping (layers - a year is one orbit of the sun, a day one rotation on the earths' axis)
% - colocated as opposed to geo-distributed
% clock synchronization literature
% - internal, external, pulse based - how have these been applied to music so far? Which work well?
% distributed systems and music
% - what distinguishes music from the rest of the clock synchronization
%   literature?
% - music (and clock sync) is not linear data! Consistency is not the primary issue. If a peer drifts or goes out of time,
%   it doesn't create merge problems. It can simply restart and all is well. Largely stateless
% - the challenge is mainly keeping tight synchronization bounds for as much time as possible
% - also practical challenges around the experience for those using the software
% - there are *some* consistency issues though. For example, synchronizing on current tempo and other transport messages

\section{Introduction}

% this intro feels like a lot of statements without citations
% but maintaining a flow is difficult

The synchronizing of events is fundamental to our perception of musical
performance. Where performers are using networked connected devices, the
challenge maintaining accurate synchronization incorporates the well studied
problem of clock synchronization from distributed systems.

As network technologies continue to grow in usage and importance to musical
performances\cite{madgwick2015simple}, it becomes increasingly important to
find common approaches to allow devices and applications to synchronize, without
reliance on expensive proprietary technology or protocols which are difficult to
implement and configure. The need to purchase or to understand such equipment
creates an unnecessary barrier to entry, potentially impeding important
contributions from musicians who lack the necessary financial or educational
resources.

Music programming environments in the academic space are well catered for with
regards to open source software, however applications in the consumer market
have tended to lag behind some of these advances. Often either the production
of music is limited to a single device, or additional devices are synchronized
using specialized hardware dedicated to synchronizing frames e.g. MIDI, SMTPE -
see Goltz\cite{goltz2018ableton} for a review of these methods.


\subsection{About Ableton Link}

Ableton Link aims to address some of these issues by supplying an open source,
permissively licensed C++ library for integration with application code. As
well as the popular digital audio workstation Ableton Live, implementations
exist for a large number of mobile applications and for many of the popular
music programming environments. In adition the following three design goals are
stated\cite{goltz2018ableton}:

\begin{itemize}
  \item Remove the restrictions of a typical master/client system
  \item Remove the requirement for initial setup
  \item Scale to a wide variety of music applications
\end{itemize}

It differs from existing approaches in that it does not rely on a master
process to propagate timing information directly. Instead nodes will establish
a session, using a reference to the start time of the oldest member of the
group even if that member is no longer present.

Clock synchronization is performed using a Kalman
filter\cite{bletsas2005evaluation} which adds a level of robustness to jitter
introduced by the network along with a more accurate reflection of the real
delay. This approach is relatively sophisticated when compared with others
using basic averaging algorithms.

Implementation is handled by application developers who are left to integrate a
small API. C++ has widespread support for integration with many popular
languages, making this viable for the majority of existing applications.

Setup for the end user is virtually transparent - network discovery takes place
automatically on all interfaces. Clock synchronization is performed
automatically for nodes joining a session and then at 60 second intervals
afterwards. Simple transport commands (start, stop) and tempo changes are also
propagated automatically by reliable broadcast.

Failures, drop outs and re-entry are all handled with a model of eventual
consistency where last-write-wins takes effect for changes in tempo and
transport state.

\subsection{Evaluating synchronization for music systems}

With the advent of distributed systems for music applications, there are
natural questions around how these should be assessed to determine
their effectiveness. Traditional criteria from distributed systems research
may be useful, particularly around the bounds synchronization, however musical
performance doesn't have the same strict requirements as traditional databases,
for example.

This means that bounds for synchronization and notions of data consistency may
be relaxed if doing so would benefit some other aspect, such as ease of setting
up.

\subsection{Fundamental problems of music synchronization}

To characterize some of the challenges more specifically, the following list
covers some of the key criteria for success:

\begin{itemize}
  \item Getting clocks in sync \cite[Chapter~6.3.2]{attiya2004distributed}
  \item Keeping them in sync (in the presence of drift or variable latency) \cite[Chapter~13]{attiya2004distributed}
  \item Network bandwidth (ensuring scalability as number of devices grows)
  \item Fault tolerance % (master availability, *musical* consistency, convergence time)
  \item Ease of setup and deployment
\end{itemize}

\section{Background research}

The use of networks in computer music is an active area of study, with much of
the research being driven by "laptop orchestras"\cite{trueman2007laptop}
centered around academic institutions. This has led to approaches centering
around the Open Sound Control
protocol\cite{wright2005open}\cite{madgwick2015simple}\cite{narveson2013landini}
as the "lingua franca" of connected musical applications, although older
methods include the use of MIDI, SMPTE and other standards (see \cite{goltz2018ableton}).

Ableton Link uses a custom network protocol but all networking is handled via
private methods within the library, meaning that application developers
choosing to implement it do not need to concern themselves with network level code.

\subsection{Finding bounds - the limits of human perception}

In order to determine some lower and upper bounds on the level of synchrony
required, data gathered around musical perception will be useful. While the
consensus is not complete\cite{greeff2016influence}, one can assume that
musical performers can tolerate up to 40ms of latency between sources. This is
of course dependent on the individual performer as well as other factors such
as the frequency domain of the sound.

In addition to delays that impact live performers, any delay introduces the
risk of comb filtering effects on sound from multiple sources. This occurs when
the frequencies from once sound source reinforce or cancel out those from
another sound source.

Finally, the accurate synchronization of multiple speakers is essential for the
use of sound spatialization effects, such as stereo panning or surround sound.

With these in mind, the lower bound would ideally be zero (perfect
synchronization) but this is unlikely in practice. Delays of 0.05ms could
theoretically introduce comb filtering effects in the audible frequency range
at 10kHz\cite{lester2007effects} so this may offer a more practical lower bound.

In terms of an upper bound, 40ms of delay would appear to be the upper limit in
terms of the impact on the majority of listeners. Where human performers are involved,
this is more likely to impact their output so a lower figure of 20ms may be
more appropriate.

\subsection{Prior testing approaches}

LANdini\cite{narveson2013landini} was designed for use with a specific laptop
orchestra in mind and was "tested" in rehearsal and performance. Functional
testing is also described in their paper regarding message delivery and
bandwidth usage.

A more rigorous testing scheme is proposed in the development of
PiGMI\cite{Oda2016} (The Raspbery Pi Global Metronome) in which metronome
pulses were produced by the synchronized device at 120 pulses per minute for a
duration of 30 minutes. The output is then recorded into separate channels on a
sound card and later analyzed to determine offsets. In addition to this,
commercial drum machines were also measured, synchronized using MIDI time
clock, to add a benchmark.

This method of testing allows for excellent accuracy measuring "time at
speaker" which is arguably the most realistic metric available; However, the
recording and analysis appear to be labour intensive and the reliance on
physical hardware makes replication of results more challenging.

A similar result may be achieved using virtual audio inputs on a single machine
e.g. using the Soundflower application on OS X, however these are subject to
their own sources of latency and solution that relies on the system clock would
not be able to detect drift relative to real time if additional reference time
sources are not used also.

\section{Criteria and design}

The testing approach taken in this work chooses not to measure the audio output
but instead analyzes the reported times from each of the application nodes for
each beat at the current tempo.  This allows for simple reproduction of results
and a more flexible way to test different failure modes and network conditions.

This is achieved using the Jepsen framework\cite{jepsen}, which was
originally designed to test safety guarantees under fault injection, however it
is also flexible enough to test systems with less strict guarantees.

The framework starts 5 instances running a simple Ableton Link application
inside Linux containers using Docker. In addition to these, a control node is
started which invokes and records operations against the existing nodes.
Primarily this is concerned with read and write operations against the tempo
parameter of the Link session. Each node is also logging it's status at each
beat. These are then analyzed following the test to calculate the offsets and
convergence of events following changes in tempo.

The Jepsen framework also allows the control node to introduce faults into the
virtual network between the nodes. Under initial conditions they are all
connected, however links between nodes can be cut (packets between nodes all
dropped) to form different topologies. Variable or fixed latency can also be
introduced. These take place via a process called a "nemesis" in the framework
terminology.

\subsection{Consistency in a musical context}

The introduction of faults and pertubations to the network can cause nodes to
receive updates late, resulting in divergence for the beat markers in the
session. Provided the network remains connected these will recover eventually.

This leads to an interesting question of how to quantify the divergence and the
resulting effect on the music. This work puts forward that Ableton Link and
similar eventually consistent systems in future should measure "musical
consistency" as the time spent in agreement relative to the length of the
overall session.

Another key metric would appear to be the duration for periods of divergence.
If these can be minimized the resulting adverse impact on the musical
performance can be limited.

\subsection{Synchronization Accuracy versus Bandwidth}

In order to combat clock drift, some synchronization systems will allow a high
frequency of synchronization events to ensure greater accuracy. This comes at
the expense of more messages being sent over the network. For example, the
LANdini project opts for 3 synchronization messages (pings) every second.

For comparison, recent work by Geng et al\cite{geng2018} achieves
synchronization on the order of 10s of nanoseconds within datacenters, however
around 5 MBits/s of bandwidth is used to achieve this result.

Given that Ableton Link targets a mass market, the use of consumer grade
routers should be assumed. This makes it important to minimize the number of
messages sent by the protocol to avoid overloading the hardware, while striving
for the figures set out in tolerable latency above.

\subsection{Use of Jepsen, Docker and log parsing}

While Jepsen is primarily designed to exercise safety guarantees of distributed
systems under partitions, the flexibility allows for different kinds of tests
to be performed. This work opts mainly to use Jepsen to handle the running of
tests and fault injection, whereas the analysis of results takes place in
separate scripts.

The use of Docker containers allows straightforward reproducbility on
development machines. One possible application would be in a continous
integration testing environment so that changes to the Link codebase could be
exercised against these tests automatically.

As the Link protocol favours transparent setup over configuration, there is a
limited about of information available regarding the state of the sesssion by
default. For these tests the debug logging is enabled and these logs are then
parsed following the test to produce the output in the results section below.

\subsection{Topologies and latencies}

As part of the stated aim of ease of setup, the Ableton Link protocol performs
service discovery and message broadcast on all network interfaces to ensure
that all connected nodes are able to join the same session. For example, nodes
using a LAN may also see other nodes on a WiFi network provided that at least
one node existed that was connected to both.

This feature introduces the prospect of topologies other than a connected
network in practice. Jepsen allows different topologies to be defined and
implemented during the test procedure. As Link operates using reliable
broadcast (retransmitting the state of the session on all interfaces when a
valid state update is received) the topologies may be ranked according to the
length of the maximum path length for an update to be propagated.

\begin{itemize}
  \item Connected (easy)
  \item Bridge (medium)
  \item Line (hard)
\end{itemize}

Another test condition concerns the performance of the protocol under small,
constant network delays. In the tests below this is defined as 100ms delays,
which for clock synchronization messages results in a round trip time delta of
+200ms.

Finally the protocol is exercised in the presence of large, variable or uneven
network delays. The timings for these are included on the charts below.

\section{Experimental results}

Results here, maybe detailed graphs in appendix

\section{Analysis}

Resilient to different network conditions. Usually converges.

Talk about indeterminate divergence

\subsection{Identifying and triggering invalid states}

\section{Further work}

Generalize the work to other solutions.

Create CI testing for the Link library.

Suggest improvements to Link - Google nanosecond precision paper. Potential
uncertainty window around synchronization events. Configuration/visibility of
session information.

Integration into Sonic Pi. Windows compatibility.

\section{Conclusions}

\bibliographystyle{plain}
\bibliography{bibliography}

\appendix
\section{Self assessment}
\section{Running the tests}
\section{Configuring Jepsen to support other systems and tests}
\section{Code listings}
\section{Professional issues}
\section{Acknowledgements}

Open source code
Colleagues at Heroku
Salesforce for funding
Gregory and staff at RHUL
Em

\end{document}
