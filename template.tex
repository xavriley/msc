\documentclass[11pt]{article} % do not change this line
\input{BigDataStyle.txt}      % do not change this line
\usepackage{amsmath,amsfonts,amssymb,amsthm,latexsym,graphicx}

\emergencystretch=5mm
\tolerance=400
\allowdisplaybreaks[4]

\theoremstyle{plain}
\newtheorem{theorem}{Theorem}[section]
\newtheorem{proposition}[theorem]{Proposition}
\newtheorem{corollary}[theorem]{Corollary}
\newtheorem{lemma}[theorem]{Lemma}
\newtheorem{problem}[theorem]{Problem}

\theoremstyle{definition}
\newtheorem*{remark}{Remark}

\title{Breaking Links: Evaluating distributed time synchronization for musical applications using Ableton Link}
\author{Xavier Riley}

\newcommand{\Programme}{Distributed and Networked Systems}

\begin{document}
\maketitle

\declaration

% assessment criteria
% - bounds on synchronization
% - message complexity /bandwidth
% - difficulty of setup/integration
% - failure modes
% - resilience to drift
% - topology / spof?

% Ableton Live
% - aims, tradeoffs, assumptions (reliable network)
% - works via multicast, discovery, exchanging timelines
% - fault tolerance concepts (sending on all changes)
% Can it be improved upon without violating simplicity? To provide:
% - resilience to lost messages (reliable broadcast) - does it already do this?
% - resistance to crash faults (simple consensus) - protocol assumes a shared network - not sure this makes sense
% - improvements in synchronization bounds - what are the current limits? - are they perceptible? (within 10ms)
% - other things I haven't considered yet

% Possible experiment criteria
% We can't measure things like dropped writes or stale reads so what are the
% criteria for a music system? Draw comparisons with gossip analysis -
% convergence time, stability, resilience to pertubations. Also check how clock
% sync literature handles this.
% The main thing is that the tempo stays in sync for as much time as possible

% further ideas - even better internal sync (at the cost of bandwidth) - Google paper
% resettable oscillators - more naturally maps onto how humans synchronize perhaps
% options for geo-distributed time synchronization
% extensions beyond music to other time sensitive domains

% What your project *must* contain
% motivations and original aims including how this work may help in future career
% assessment inc. self evaluation. How did it go? What did you do right or
%      wrong? What have you learnt about planning and execution? Where next?

% should contain
% - abstract
% - introduction
% - background research (literature survey/review)
% - either

% sofware product - software engineering method, requirements analysis, design,
% implementation, testing - also user or installation manual

% theoretical - development of theory, inc small programs, explanation of
% algorithms, descriptions of hard theory, results, analysis

% experimental - experimental results, analysis, conclusions

% - professional issues (as appendix - c. 1000 words)
% - self assessment
% - bibliography
% - any code, including instructions on how to run
% - other - layout diagrams, sample output, program listing

\begin{abstract}
  With the recent explosion of connected musical devices, the challenge of
  making these play "in time" with each other mounts against application
  developers and device manufacturers. Ableton Link\cite{goltz2018ableton} aims
  to provide robust, resilient musical synchronization using principles from
  distributed systems programming, in contrast to previous master/slave
  approaches. However the evaluation criteria for such a system are not well
  represented in the existing literature, with particular reference to a
  musical context.  The following presents a system for empirical testing of
  the Ableton Link library using the Jepsen\cite{jepsen} testing framework,
  along with a set of criteria for evaluting similar libraries that may be
  developed in future.
\end{abstract}

% Motivations
%
% Artistic output but also, music industry is large and $$$
% requirements of a musical performance
% - ensemble
% - timekeeping (layers - a year is one orbit of the sun, a day one rotation on the earths' axis)
% - colocated as opposed to geo-distributed
% clock synchronization literature
% - internal, external, pulse based - how have these been applied to music so far? Which work well?
% distributed systems and music
% - what distinguishes music from the rest of the clock synchronization
%   literature?
% - music (and clock sync) is not linear data! Consistency is not the primary issue. If a peer drifts or goes out of time,
%   it doesn't create merge problems. It can simply restart and all is well. Largely stateless
% - the challenge is mainly keeping tight synchronization bounds for as much time as possible
% - also practical challenges around the experience for those using the software
% - there are *some* consistency issues though. For example, synchronizing on current tempo and other transport messages

\section{Introduction}

% this intro feels like a lot of statements without citations
% but maintaining a flow is difficult

The synchronizing of events is fundamental to our perception of musical
performance. Where performers are using networked connected devices, the
challenge incorporates the well studied problem of clock synchronization from
distributed systems.

As network technologies continue to grow in usage and importance to musical
performances\cite{madgwick2015simple}, it becomes increasingly important to
find common approaches to allow devices and applications to synchronize without
reliance on proprietary protocols. Music programming environments in the
academic space are well catered for in this regard, however applications in the
consumer market have tended to lag behind advances in technology. Either the
production of music is limited to a single device, or additional devices are
synchronized using specialized hardware dedicated to synchronizing frames.

\subsection{Fundamental problems of music synchronization}

* getting clocks in sync
* keeping them in sync (drift, latency)
* network bandwidth (commodity network hardware, high number of devices)
* fault tolerance (master availability, *musical* consistency, convergence time)
* ease of setup and deployment

\subsection{About Ableton Link}

Ableton Link aims to solve some of these issues by supplying an open source,
permissively licensed C++ library for integration with application code. As
well as the popular digital audio workstation Ableton Live, implementations
exist for a large number of mobile applications and for many of the popular
music programming environments.

It differs from existing approaches in that it does not rely on a master
process to propagate timing information directly. Instead nodes will establish
a session, using a reference to the start time of the oldest member of the
group even if that member is no longer present.

Clock synchronization is performed using a Kalman filter\cite{} which adds a
level of robustness to jitter introduced by the network along with a more
accurate reflection of the real delay. This approach is relatively
sophisticated when compared with others using Cristian algorithm.

Integration is handled by application developers who are left to integrate a
small API. C++ has widespread support for integration with many popular
languages, making this viable for the majority of existing applications.

Setup for the end user is virtually transparent - network discovery takes place
automatically on all interfaces. Clock synchronization is performed
automatically for nodes joining a session and then at 60 second intervals
afterwards. Simple transport commands (start, stop) and tempo changes are also
propagated automatically by reliable broadcast.

Failures, drop outs and re-entry are all handled with a model of eventual
consistency where last-write-wins takes effect for changes in tempo and
transport state.

\section{Background research}

The use of networks in computer music is an active area of study, with much of
the research being driven by "laptop orchestras"\cite{trueman2007laptop}
centered around academic institutions. This has led to approaches centering
around the Open Sound Control
protocol\cite{wright2005open}\cite{madgwick2015simple}\cite{narveson2013landini}
as the "lingua franca" of connected musical applications, although older
methods include the use of MIDI, SMPTE and other standards - see
Goltz\cite{goltz2018ableton} for a review of these.

\subsection{Prior testing approaches}

% how to test it? One option - get several devices and record into same soundcard
% has the benefit of being a realistic test, easier to get setup
% other option - use soundflower and write to a multichannel wav - maybe more reproducible
% more scientific - setup jepsen and log timestamps

* in production/performance (Landini)
* via hardware recording to sound file (PiGMI + Landini)

% experimental - experimental results, analysis, conclusions
\section{Criteria and design}

\subsection{Consistency in a musical context}

Convergence time as a percentage of the session

Longest time for period of divergence.

\subsection{Accuracy versus Bandwidth}
\subsection{Use of Jepsen, Docker and log parsing}

Jepsen desgined to test linearizability through fault injection, however it is flexible.

Docker allows straightforward reproducbility on development machines or in CI testing.

Use of debug logs in Link

\subsection{Topologies and latencies}

Easy (connected, zero delay)
Medium (Ring? bridge? zero delay)
Hard (Line, zero delay)

ditto small delay

ditto large, variable or uneven delay

\section{Experimental results}

Results here, maybe detailed graphs in appendix

\section{Analysis}

Resilient to different network conditions. Usually converges.

\subsection{Identifying and triggering invalid states}

\section{Further work}

Generalize the work to other solutions.

Create CI testing for the Link library.

Suggest improvements to Link - Google nanosecond precision paper. Potential
uncertainty window around synchronization events. Configuration/visibility of
session information.

Integration into Sonic Pi. Windows compatibility.

\section{Conclusions}

\bibliographystyle{plain}
\bibliography{bibliography}

\appendix
\section{Self assessment}
\section{Running the tests}
\section{Configuring Jepsen to support other systems and tests}
\section{Code listings}
\section{Professional issues}
\section{Acknowledgements}

Open source code
Colleagues at Heroku
Salesforce for funding
Gregory and staff at RHUL
Em

\end{document}
